\documentclass[conference,a4paper,10pt, oneside,final]{tfmpd}
%\usepackage[latin1]{inputenc}   % caracteres especiales (acentos, eñes)
%\usepackage[spanish]{babel}     % varias definiciones para el español
\usepackage[utf8x]{inputenc}
\usepackage{graphicx}           % inserción de graficos
\usepackage{float} %paquete para que ponga las imagenees donde yo quiero, no donde quiere el muy hdp de latex
%la posición de la imagen, con H (mayúscula!!)

\usepackage{amssymb}
\usepackage{amsfonts}
\usepackage{amsmath}

\begin{document}

\title{Técnicas de Eseganografía en señales de habla.}

\author{Darío A. Villarreal,
        Esteban Zeller y
        Matías A. Eberhardt\\
\textit{Trabajo Práctico Final de Procesamiento Digital de Señales, II-FICH-UNL.}}

\markboth{Trabajo Práctico Final de Procesamiento Digital de Señales}{}

\maketitle

\begin{abstract}
En este trabajo haremos una breve clasificación de las técnicas de esteganografía existentes y su diferenciación con otras disciplinas de protección de datos digitales. Luego decribiremos e implementaremos dos de dichas técnicas: Modificación del Bit Menos Significativo (LSB, Least Significant Bit) en el dominio temporal y en el dominio frecuencial se ocultará la información en los coeficientes de la Tranformada Wavelet Discreta. Finalmente evaluaremos los resultados obtenidos mediante técnicas objetivas y subjetivas.
\end{abstract}

\begin{keywords}
esteganografía, data hidding, watermarking
\end{keywords}

\section{Introducción}
   

\PARstart{L}{a} esteganografía es una disciplina que se basa en ocultar mensajes u objetos dentro de otros llamados portadores de modo que su inclusión pase desapercibida. Básicamente explota las limitaciones de la percepción humana, ya que nuestros sentidos presentan límites para percibir pequeñas alteraciones en las señales.

\section{Disciplinas de protección de datos digitales}
Encriptación vs Esteganografia vs Watermaking

\section{Clasificación de métodos esteganográficos}
\section{Implementación en el dominio temporal}
\section{Implementación en el dominio frecuencial}
\section{Evaluación de resultados}
\section{Conclusiones}

\nocite{*}
\bibliographystyle{tfmpd}
\bibliography{tfmpd}

\end{document}
